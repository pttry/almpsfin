\documentclass[12pt]{article}
\expandafter\def\csname
ver@l3backend.sty\endcsname{}

\usepackage{geometry}
\usepackage[hidelinks]{hyperref}
\usepackage{setspace}
\usepackage{xcolor}
\usepackage{graphicx}
\usepackage[utf8]{inputenc}
\usepackage{fontspec}
\usepackage{l3backend}
\usepackage{polyglossia}
\usepackage{testhyphens}
\usepackage{pgffor}

\setmainlanguage{finnish}
\addto\captionsfinnish{\renewcommand{\figurename}{Kuvio}}

\newcommand{\datalahde}[1]{\input{../plots/eurostat_statfin/#1_selite.txt}\unskip} 
\newcommand{\captionselite}[1] {\textit{\footnotesize{#1}}}


 \geometry{
 a4paper,
 % total={165mm,230mm},
 left=25mm,
 top=25mm,
 right=25mm
 }
 
\begin{document}

\noindent \textbf{Euroopan komission ja Tilastokeskuksen työvoimapalvelujen luokitus} \par
\vspace{0.5cm}

\noindent Juho Alasalmi{\par}
\noindent \href{mailto: juho.alasalmi@ptt.fi}{\textcolor{blue}{juho.alasalmi@ptt.fi}}{\par}
\noindent Pellervon Taloustutkimus {\par}

\vspace{0.5cm}
\setstretch{1.2}

Euroopan komissio pyrkii tietoja kerätessään luomaan aineiston, joka on vertailukel-poinen eri valtioiden välillä. Työ- ja elinkeinoministeriön on tietojaan luovuttaessaan sovitettava Suomessa käytössä olevat työvoimapalvelut Euroopan komission jaotteluohjeisiin. Eri palveluiden määritelmät Euroopan komission aineistossa ovat hieman karkeammat ja osaltaan erilaiset kuin TE-toimistojen työvoimapalvelujen rekisteröinnissä ja yksilötason rekisteriaineistoissa.

Euroopan komission aineiston luokittelu ei myöskään sisällä tietoa sen yhteydestä kansallisiin, esimerkiksi Työnvälitystilaston käyttämään luokitukseen. Yhdistämme Euroopan komission aineiston työvoimapalvelujen luokituksen Tilastokeskuksen StatFin-palvelun kautta saataviin työnvälitystilaston työvoimapalvelujen luokituksiin vertailemalla näiden aineistojen tietoja palveluissa olleiden, aloittaneiden ja lopettaneiden määristä. Käytännössä tarkastelemme mitkä StatFin-palvelussa käytössä olevat työvoimapalvelujen koodit ryhmänä tuottavat samat osallistuneiden, aloittaneiden ja lopettaneiden määrät kuin kukin koodi Euroopan komission aineistossa.

Taulukko \ref{tbl:sldkjf23} esitää avaimen joka antaa kullekin Euroopan komission työvoimapalvelukoodille StatFin-palvelun vastaavat työvoimapalvelujen koodit. Kuviot 1-16 esittävät kuinka koodien ryhmittelyt vastaavat toisinaan kyseisten työvoimapalvelujen osallistujien, aloittaneiden ja lopettaneiden määrien mukaan.

\vspace{1cm}
\noindent Euroopan komission aineisto: \\ \footnotesize \href{https://webgate.ec.europa.eu/empl/redisstat/databrowser/product/page/LMP_EXPME$FI}{\textcolor{blue}{https://webgate.ec.europa.eu/empl/redisstat/databrowser/product/page/LMP\textunderscore EXPME\$FI}} \normalsize

\vspace{1cm} 
\noindent StatFin-palvelu: \\ \footnotesize\href{https://pxnet2.stat.fi/PXWeb/pxweb/fi/StatFin/}{\textcolor{blue}{https://pxnet2.stat.fi/PXWeb/pxweb/fi/StatFin/}}\normalsize

\vspace{1cm}
\noindent Luokitusavain .rda-tiedostona ja tämän analyysin koodit: \\  \footnotesize\href{https://github.com/pttry/almpsfin}{\textcolor{blue}{https://github.com/pttry/almpsfin}}\normalsize
\begin{table}[b]
\footnotesize
\centering
\begin{tabular}{l|l|l|l}
LMP TYPE & code statfin & lmp name fin & palveluluokka \\ \hline

21\textunderscore FI6 & 0 & Työvoimakoulutus & muu \\ 21\textunderscore FI6 & 1 & Työvoimakoulutus & muu \\ 7\textunderscore FI11 & 65 & Starttiraha & tyollistaminen \\ 7\textunderscore FI11 & 62 & Starttiraha & tyollistaminen \\ 7\textunderscore FI55 & 61 & Starttiraha ei-työttömille & tyollistaminen \\ 24\textunderscore FI7 & 51 & Oppisopimuskoulutus työttömille & tyollistaminen \\ 24\textunderscore FI7 & 55 & Oppisopimuskoulutus työttömille & tyollistaminen \\ 24\textunderscore FI7 & 56 & Oppisopimuskoulutus työttömille & tyollistaminen \\ 24\textunderscore FI7 & 58 & Oppisopimuskoulutus työttömille & tyollistaminen \\ 24\textunderscore FI7 & 68 & Oppisopimuskoulutus työttömille & tyollistaminen \\ 24\textunderscore FI7 & 87 & Oppisopimuskoulutus työttömille & tyollistaminen \\ 24\textunderscore FI7 & 88 & Oppisopimuskoulutus työttömille & tyollistaminen \\ 24\textunderscore FI7 & 89 & Oppisopimuskoulutus työttömille & tyollistaminen \\ 41\textunderscore FI10 & 80 & Palkkatuki yksityiselle & tyollistaminen \\ 41\textunderscore FI10 & 81 & Palkkatuki yksityiselle & tyollistaminen \\ 41\textunderscore FI10 & 6T & Palkkatuki yksityiselle & tyollistaminen \\ 41\textunderscore FI10 & 6V & Palkkatuki yksityiselle & tyollistaminen \\ 41\textunderscore FI10 & 60 & Palkkatuki yksityiselle & tyollistaminen \\ 41\textunderscore FI10 & 69 & Palkkatuki yksityiselle & tyollistaminen \\ 6\textunderscore FI9 & 54 & Palkkatuki kunnalle & tyollistaminen \\ 6\textunderscore FI9 & 57 & Palkkatuki kunnalle & tyollistaminen \\ 6\textunderscore FI9 & 5T & Palkkatuki kunnalle & tyollistaminen \\ 6\textunderscore FI9 & 5V & Palkkatuki kunnalle & tyollistaminen \\ 6\textunderscore FI9 & 53 & Palkkatuki kunnalle & tyollistaminen \\ 6\textunderscore FI9 & 50 & Palkkatuki kunnalle & tyollistaminen \\ 6\textunderscore FI8 & 40 & Valtiolle työllistäminen & tyollistaminen \\ 6\textunderscore FI8 & 41 & Valtiolle työllistäminen & tyollistaminen \\ 11\textunderscore FI59 & 3 & Työnhakuvalmennus & valmennus \\ 11\textunderscore FI60 & 5 & Uravalmennus & valmennus \\ 11\textunderscore FI61 & 4 & Työhönvalmennus & valmennus \\ 21\textunderscore FI63 & 36 & Koulutuskokeilu & kokeilu \\ 22\textunderscore FI62 & 20 & Työkokeilu & kokeilu \\ 22\textunderscore FI62 & 21 & Työkokeilu & kokeilu \\ 22\textunderscore FI66 & 22 & Rekrytointikokeilu & kokeilu \\ 6\textunderscore FI36 & 05 & Kuntouttava työtoiminta & muu \\ 21\textunderscore FI17 & 06 & Omaehtoinen opiskelu työttömyysetuudella & muu \\ 43\textunderscore FI16 & 02 & Työnvuorottelu & muu

\end{tabular}
\caption{Työvoimapalvelujen luokittelu Euroopan komission ja Tilastokeskuksen aineistoissa.}
\label{tbl:sldkjf23}
\end{table}

\foreach \palvelu in {Työvoimakoulutus, Starttiraha, Starttiraha ei-työttömille, Oppisopimuskoulutus työttömille, Palkkatuki yksityiselle, Palkkatuki kunnalle, Valtiolle työllistäminen, Työnhakuvalmennus, Uravalmennus, Työhönvalmennus, Koulutuskokeilu, Työkokeilu, Rekrytointikokeilu, Kuntouttava työtoiminta, Omaehtoinen opiskelu työttömyysetuudella, Työnvuorottelu} {

\begin{figure}[b]
\centering
\includegraphics[scale = 0.9]{../plots/eurostat_statfin/\palvelu.pdf}
\caption{\palvelu, palvelussa olleet, aloittaneet ja lopettaneet Euroopan komission ja Tilastokeskuksen aineistoissa. \captionselite{ \protect \datalahde{\palvelu}}}
   \label{fig:kdieksl}
\end{figure}
}

\end{document}